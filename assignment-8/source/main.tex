\documentclass{article}
\usepackage[utf8]{inputenc}
\usepackage{karnaugh-map}

\title{DLD  Assignment 8}
\author{Mahesh Vasimalla}
\date{January 2021}

\begin{document}

\maketitle

\section{Boolean equation}
\begin{equation}
F\;=\;\overline{p}\;q\;+\;p\;+\;\overline{q}
\end{equation}

\section{K-Map}
\begin{figure}[h]
    \centering
    \begin{karnaugh-map}[2][2][1][][]
    \minterms{1,2}
    \autoterms[0]
    \implicant{0}{0}
    \implicant{3}{3}
    \draw[color=black, ultra thin] (0, 2) --
    node [pos=0.7, above right, anchor=south west] {$p$} 
    node [pos=0.7, below left, anchor=north east] {$q$} 
    ++(135:1);
\end{karnaugh-map}
    \caption{K-map for eq.(1)} 
    \label{fig:my_label}
\end{figure}

Using the above K-map, we can find the POS form as given below
    
\section{Product of Sums form}    
\begin{equation}
    F\;=\;(p\;+\;q)(\overline{p}\;+\;\overline{q})
\end{equation}    


\end{document}
